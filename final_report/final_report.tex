%% paper_template.tex is a modification of:
%% bare_conf.tex 
%% V1.2
%% 2002/11/18
%% by Michael Shell
%% mshell@ece.gatech.edu
%% 
%% This is a skeleton file demonstrating the use of IEEEtran.cls 
%% (requires IEEEtran.cls version 1.6b or later) with an IEEE conference paper.
%% 
%% Support sites:
%% http://www.ieee.org
%% and/or
%% http://www.ctan.org/tex-archive/macros/latex/contrib/supported/IEEEtran/ 
%%
%% This code is offered as-is - no warranty - user assumes all risk.
%% Free to use, distribute and modify.

% *** Authors should verify (and, if needed, correct) their LaTeX system  ***
% *** with the testflow diagnostic prior to trusting their LaTeX platform ***
% *** with production work. IEEE's font choices can trigger bugs that do  ***
% *** not appear when using other class files.                            ***
% Testflow can be obtained at:
% http://www.ctan.org/tex-archive/macros/latex/contrib/supported/IEEEtran/testflow


% Note that the a4paper option is mainly intended so that authors in
% countries using A4 can easily print to A4 and see how their papers will
% look in print. Authors are encouraged to use U.S. letter paper when 
% submitting to IEEE. Use the testflow package mentioned above to verify
% correct handling of both paper sizes by the author's LaTeX system.
%
% Also note that the "draftcls" or "draftclsnofoot", not "draft", option
% should be used if it is desired that the figures are to be displayed in
% draft mode.
%
% This paper can be formatted using the % (instead of conference) mode.
%++++++++++++++++++++++++++++++++++++++++++++++++++++++
%\documentclass[conference]{IEEEims} % Modified for MTT-IMS
%\documentclass[conference]{IMSTemplate}
\documentclass[conference]{IEEEtran}
%++++++++++++++++++++++++++++++++++++++++++++++++++++++
% If the IEEEtran.cls has not been installed into the LaTeX system files, 
% manually specify the path to it:
% \documentclass[conference]{../sty/IEEEtran} 


% some very useful LaTeX packages include:

%\usepackage{cite}      % Written by Donald Arseneau
                        % V1.6 and later of IEEEtran pre-defines the format
                        % of the cite.sty package \cite{} output to follow
                        % that of IEEE. Loading the cite package will
                        % result in citation numbers being automatically
                        % sorted and properly "ranged". i.e.,
                        % [1], [9], [2], [7], [5], [6]
                        % (without using cite.sty)
                        % will become:
                        % [1], [2], [5]--[7], [9] (using cite.sty)
                        % cite.sty's \cite will automatically add leading
                        % space, if needed. Use cite.sty's noadjust option
                        % (cite.sty V3.8 and later) if you want to turn this
                        % off. cite.sty is already installed on most LaTeX
                        % systems. The latest version can be obtained at:
                        % http://www.ctan.org/tex-archive/macros/latex/contrib/supported/cite/

%\usepackage{graphicx}  % Written by David Carlisle and Sebastian Rahtz
                        % Required if you want graphics, photos, etc.
                        % graphicx.sty is already installed on most LaTeX
                        % systems. The latest version and documentation can
                        % be obtained at:
                        % http://www.ctan.org/tex-archive/macros/latex/required/graphics/
                        % Another good source of documentation is "Using
                        % Imported Graphics in LaTeX2e" by Keith Reckdahl
                        % which can be found as esplatex.ps and epslatex.pdf
                        % at: http://www.ctan.org/tex-archive/info/
% NOTE: for dual use with latex and pdflatex, instead load graphicx like:
%\ifx\pdfoutput\undefined
%\usepackage{graphicx}
%\else
%\usepackage[pdftex]{graphicx}
%\fi
%+++++++++++++++++++++++++++++++++++++++++++
% Added to commands
\input epsf
\usepackage{graphicx}
\usepackage{hyperref}
\newcommand{\code}[1]{\texttt{\detokenize{#1}}}

%+++++++++++++++++++++++++++++++++++++++++++
% However, be warned that pdflatex will require graphics to be in PDF
% (not EPS) format and will preclude the use of PostScript based LaTeX
% packages such as psfrag.sty and pstricks.sty. IEEE conferences typically
% allow PDF graphics (and hence pdfLaTeX). However, IEEE journals do not
% (yet) allow image formats other than EPS or TIFF. Therefore, authors of
% journal papers should use traditional LaTeX with EPS graphics.
%
% The path(s) to the graphics files can also be declared: e.g.,
% \graphicspath{{../eps/}{../ps/}}
% if the graphics files are not located in the same directory as the
% .tex file. This can be done in each branch of the conditional above
% (after graphicx is loaded) to handle the EPS and PDF cases separately.
% In this way, full path information will not have to be specified in
% each \includegraphics command.
%
% Note that, when switching from latex to pdflatex and vice-versa, the new
% compiler will have to be run twice to clear some warnings.


%\usepackage{psfrag}    % Written by Craig Barratt, Michael C. Grant,
                        % and David Carlisle
                        % This package allows you to substitute LaTeX
                        % commands for text in imported EPS graphic files.
                        % In this way, LaTeX symbols can be placed into
                        % graphics that have been generated by other
                        % applications. You must use latex->dvips->ps2pdf
                        % workflow (not direct pdf output from pdflatex) if
                        % you wish to use this capability because it works
                        % via some PostScript tricks. Alternatively, the
                        % graphics could be processed as separate files via
                        % psfrag and dvips, then converted to PDF for
                        % inclusion in the main file which uses pdflatex.
                        % Docs are in "The PSfrag System" by Michael C. Grant
                        % and David Carlisle. There is also some information 
                        % about using psfrag in "Using Imported Graphics in
                        % LaTeX2e" by Keith Reckdahl which documents the
                        % graphicx package (see above). The psfrag package
                        % and documentation can be obtained at:
                        % http://www.ctan.org/tex-archive/macros/latex/contrib/supported/psfrag/

%\usepackage{subfigure} % Written by Steven Douglas Cochran
                        % This package makes it easy to put subfigures
                        % in your figures. i.e., "figure 1a and 1b"
                        % Docs are in "Using Imported Graphics in LaTeX2e"
                        % by Keith Reckdahl which also documents the graphicx
                        % package (see above). subfigure.sty is already
                        % installed on most LaTeX systems. The latest version
                        % and documentation can be obtained at:
                        % http://www.ctan.org/tex-archive/macros/latex/contrib/supported/subfigure/

%\usepackage{url}       % Written by Donald Arseneau
                        % Provides better support for handling and breaking
                        % URLs. url.sty is already installed on most LaTeX
                        % systems. The latest version can be obtained at:
                        % http://www.ctan.org/tex-archive/macros/latex/contrib/other/misc/
                        % Read the url.sty source comments for usage information.

%\usepackage{stfloats}  % Written by Sigitas Tolusis
                        % Gives LaTeX2e the ability to do double column
                        % floats at the bottom of the page as well as the top.
                        % (e.g., "\begin{figure*}[!b]" is not normally
                        % possible in LaTeX2e). This is an invasive package
                        % which rewrites many portions of the LaTeX2e output
                        % routines. It may not work with other packages that
                        % modify the LaTeX2e output routine and/or with other
                        % versions of LaTeX. The latest version and
                        % documentation can be obtained at:
                        % http://www.ctan.org/tex-archive/macros/latex/contrib/supported/sttools/
                        % Documentation is contained in the stfloats.sty
                        % comments as well as in the presfull.pdf file.
                        % Do not use the stfloats baselinefloat ability as
                        % IEEE does not allow \baselineskip to stretch.
                        % Authors submitting work to the IEEE should note
                        % that IEEE rarely uses double column equations and
                        % that authors should try to avoid such use.
                        % Do not be tempted to use the cuted.sty or
                        % midfloat.sty package (by the same author) as IEEE
                        % does not format its papers in such ways.

%\usepackage{amsmath}   % From the American Mathematical Society
                        % A popular package that provides many helpful commands
                        % for dealing with mathematics. Note that the AMSmath
                        % package sets \interdisplaylinepenalty to 10000 thus
                        % preventing page breaks from occurring within multiline
                        % equations. Use:
%\interdisplaylinepenalty=2500
                        % after loading amsmath to restore such page breaks
                        % as IEEEtran.cls normally does. amsmath.sty is already
                        % installed on most LaTeX systems. The latest version
                        % and documentation can be obtained at:
                        % http://www.ctan.org/tex-archive/macros/latex/required/amslatex/math/



% Other popular packages for formatting tables and equations include:

%\usepackage{array}
% Frank Mittelbach's and David Carlisle's array.sty which improves the
% LaTeX2e array and tabular environments to provide better appearances and
% additional user controls. array.sty is already installed on most systems.
% The latest version and documentation can be obtained at:
% http://www.ctan.org/tex-archive/macros/latex/required/tools/

% Mark Wooding's extremely powerful MDW tools, especially mdwmath.sty and
% mdwtab.sty which are used to format equations and tables, respectively.
% The MDWtools set is already installed on most LaTeX systems. The lastest
% version and documentation is available at:
% http://www.ctan.org/tex-archive/macros/latex/contrib/supported/mdwtools/


% V1.6 of IEEEtran contains the IEEEeqnarray family of commands that can
% be used to generate multiline equations as well as matrices, tables, etc.


% Also of notable interest:

% Scott Pakin's eqparbox package for creating (automatically sized) equal
% width boxes. Available:
% http://www.ctan.org/tex-archive/macros/latex/contrib/supported/eqparbox/



% Notes on hyperref:
% IEEEtran.cls attempts to be compliant with the hyperref package, written
% by Heiko Oberdiek and Sebastian Rahtz, which provides hyperlinks within
% a document as well as an index for PDF files (produced via pdflatex).
% However, it is a tad difficult to properly interface LaTeX classes and
% packages with this (necessarily) complex and invasive package. It is
% recommended that hyperref not be used for work that is to be submitted
% to the IEEE. Users who wish to use hyperref *must* ensure that their
% hyperref version is 6.72u or later *and* IEEEtran.cls is version 1.6b 
% or later. The latest version of hyperref can be obtained at:
%
% http://www.ctan.org/tex-archive/macros/latex/contrib/supported/hyperref/
%
% Also, be aware that cite.sty (as of version 3.9, 11/2001) and hyperref.sty
% (as of version 6.72t, 2002/07/25) do not work optimally together.
% To mediate the differences between these two packages, IEEEtran.cls, as
% of v1.6b, predefines a command that fools hyperref into thinking that
% the natbib package is being used - causing it not to modify the existing
% citation commands, and allowing cite.sty to operate as normal. However,
% as a result, citation numbers will not be hyperlinked. Another side effect
% of this approach is that the natbib.sty package will not properly load
% under IEEEtran.cls. However, current versions of natbib are not capable
% of compressing and sorting citation numbers in IEEE's style - so this
% should not be an issue. If, for some strange reason, the user wants to
% load natbib.sty under IEEEtran.cls, the following code must be placed
% before natbib.sty can be loaded:
%
% \makeatletter
% \let\NAT@parse\undefined
% \makeatother
%
% Hyperref should be loaded differently depending on whether pdflatex
% or traditional latex is being used:
%
%\ifx\pdfoutput\undefined
%\usepackage[hypertex]{hyperref}
%\else
%\usepackage[pdftex,hypertexnames=false]{hyperref}
%\fi
%
% Pdflatex produces superior hyperref results and is the recommended
% compiler for such use.



% *** Do not adjust lengths that control margins, column widths, etc. ***
% *** Do not use packages that alter fonts (such as pslatex).         ***
% There should be no need to do such things with IEEEtran.cls V1.6 and later.


\bibstyle{IEEEtran.bst}

% correct bad hyphenation here
\hyphenation{op-tical net-works semi-conduc-tor IEEEtran}
\begin{document}

% paper title
%\title{Submission Format for IMS2014 (Title in 24-point Times font)}
% If the \LARGE is deleted, the title font defaults to  24-point.
% Actually, 
% the \LARGE sets the title at 17 pt, which is close enough to 18-point.
%+++++++++++++++++++++++++++++++++++++++++++
\title{\LARGE Reddit comments analysis}
%+++++++++++++++++++++++++++++++++++++++++++
% author names and affiliations
% use a multiple column layout for up to three different
% affiliations
%+++++++++++++++++++++++++++++++++++++++++++
%\author{\authorblockN{J. Clerk Maxwell}
%\authorblockA{School of Electrical and\\Computer Engineering\\
%Somewhere Institute of Technology\\
%City, State 54321--0000\\
%Email: maxwell@curl.edu}
%\and
%\authorblockN{Michael Faraday}
%\authorblockA{(List authors on this line using 12 point Times font\\ - use a second line if necessary)\\
%Microwave Research\\
%City, State/Region, Mail/Zip Code, Country\\
%Email: homer@thesimpsons.com}
%\and
%\authorblockN{Andr\'e M. Amp\`ere \\ }
%\authorblockA{Starfleet Academy\\
%San Francisco, CA 96678-2391\\
%Telephone: (800) 555--1212\\
%Fax: (888) 555--1212}}

\author{\authorblockN{Carel Kuusk*}
\authorblockA{\authorrefmark{1}University of Tartu}}


%+++++++++++++++++++++++++++++++++++++++++++++++++++

% avoiding spaces at the end of the author lines is not a problem with
% conference papers because we don't use \thanks or \IEEEmembership


% for over three affiliations, or if they all won't fit within the width
% of the page, use this alternative format:
% 
% Another example.
%\author{\authorblockN{Michael Shell\authorrefmark{1},
%Homer Simpson\authorrefmark{2},
%James Kirk\authorrefmark{3}, 
%Montgomery Scott\authorrefmark{3} and
%Eldon Tyrell\authorrefmark{4}}
%\authorblockA{\authorrefmark{1}School of Electrical and Computer Engineering\\
%Georgia Institute of Technology,
%Atlanta, Georgia 30332--0250\\ Email: mshell@ece.gatech.edu}
%\authorblockA{\authorrefmark{2}Twentieth Century Fox, Springfield, USA\\
%Email: homer@thesimpsons.com}
%\authorblockA{\authorrefmark{3}Starfleet Academy, San Francisco, California 96678-2391\\
%Telephone: (800) 555--1212, Fax: (888) 555--1212}
%\authorblockA{\authorrefmark{4}Tyrell Inc., 123 Replicant Street, Los Angeles, California 90210--4321}}



% use only for invited papers
%\specialpapernotice{(Invited Paper)}

% make the title area
\maketitle

\begin{abstract}
Online social networks have become hugely popular in this century, with Reddit being one of the largest. Reddit is a community aggregator,
where people can join their own communities called 'subreddits'. Reddit also provides an easily accessible API and access to the data 
from the beginning of the site. This makes it possible to analyse time-evolution of early communities using a Reddit comment dataset 
scraped via the public API. The evolution of the network is characterized. 
The code and data is available on \href{https://github.com/CarelKuusk/reddit_comments/}{Github}.
\end{abstract}
\IEEEoverridecommandlockouts
\begin{keywords}
Social network analysis, Reddit, network evolution
\end{keywords}
% no keywords

% For peer review papers, you can put extra information on the cover
% page as needed:
% \begin{center} \bfseries EDICS Category: 3-BBND \end{center}
%
% for peerreview papers, inserts a page break and creates the second title.
% Will be ignored for other modes.
\IEEEpeerreviewmaketitle

%Articles 
%https://www.nature.com/articles/s41598-021-81531-x
%https://journals.sagepub.com/doi/full/10.1177/2056305118815908
%https://dl.acm.org/doi/pdf/10.1145/2567948.2576943
%https://dl.acm.org/doi/pdf/10.1145/3400806.3400814
%https://www.nature.com/articles/s41598-020-78224-2
%https://dl.acm.org/doi/fullHtml/10.1145/1400214.1400220

\section{Introduction}

With the advent of internet, social networks started to gain popularity around the turn of the century. The first major social network was SixDegrees.com, which launched in 1997. However, the rapid growth came 
with wider adoption of the internet and mobile devices, with Facebook being launched in 2004, Reddit in 2005 and Twitter in 2006 being one of the most popular ones still active today.~\cite{Howard2008}

Reddit is a massive social network, which ranks amongst the top 20 visited sites on the web by traffic. Reddit is organized in user-created communities called subreddits. Each subreddit has its own set of rules 
in addition to the global moderation policy of the Reddit website itself, a common topic and a set of volunteer moderators who are allowed to enforce the rules. Users can subscribe to subreddits, in which case the 
activity on the subreddit appears in their personal feed. Each user can post on the subscribed subreddit, and comment on posts and other comments, thus engagement under a post follows a tree structure. Users can 
upvote and downvote posts and comments, the difference between posts and comments results in the total score for the post. 

However, the early Reddit was a totally different website, being more similar to HackerNews in its organization. Users were not allowed to create their own subreddits before the start of 2008 and at first there was just 
one 'subreddit' called \textit{reddit.com}. It was not a subreddit in contemporary sense, but instead just the front page where everybody could post and comment. The first subreddits were created manually by the 
site administrators, with the oldest proper subreddit being r/features or r/nsfw created on XXX. 

However, the first subreddits, that acquired significant engagement, were r/programming (created XXX) and r/science (created XXX). This indicates the first audience Reddit was directed towards were predominantly 
male and nerdy. The early culture probably helps to explain why to this day Reddit's audience is mainly comprised of young males~\cite{Duggan2013}.

Reddit is relatively open to data acquisition compared to other large social media platforms like Facebook and Twitter. Reddit has a comprehensive API that anybody can use to acquire data from the very inception of the 
Reddit platform. Queryable entities include subreddits, users, links and comments, etc~\cite{RedditAPI}. This project used a comments dataset scraped using the publicly accessible API. 

The availability of easily accessible data even from the very beginning of a social media platform is also the main reason for choosing this project. The easily accessible and well-formatted data allows us to analyze 
how an early social networking cite evolved from a small site to the behemoth it is today. The initial plan was to analyze the evolution of different subreddits -- how users discover subreddits, why some subreddits become 
more popular, interaction between different subreddits etc.

However, during the analysis several hindering factors were discovered. First, the original Reddit did not have any user-created subreddits, and only opened 
the platform for users to create their own subreddits in 2008. This subsequently resulted in exponential growth in both the number of subreddits and the number of users (as opposed to the previous roughly linear growth). 
In and of itself it is an interesting result and confirmation of the power of user-generated content that is so widespread in todays online activities. Sadly, the exponential increase in the number of subreddits and users also 
meant that the computational capabilities accessible to the author were not sufficient to analyze such vast volumes of data. 



\section{Related work}

Possible analysis methods are provided in Cordeiro et al. (2018): snapshot analysis 
to identify more static features of the network structure and aggregate analysis (either via landmark windows or sliding windows) for 
analysing statistical properties of the graphs within a certain timeframe.

Troy Steinbauer has analyzed the overall properties of the Reddit network in 2011. He identified that for example related subreddit distribution was 
following the expected power-law distribution.

The behaviour of individual users was analyzed by Thukral et al in 2018. The analysis was partly done on the 2008 and partly on the 2014-2015 period. 
They discovered, that based on the commenting and posting patterns, the users can be clustered into three types based on when most of their contributions are made 
(either within a short period from sign-up, stably, or more actively after a long hibernation). Also they discovered a clear separation between commenters and posters.


\section{Dataset}

The dataset consists of all of Reddit's comments from 2005 to 2008 (included), that is from the inception of the Reddit. This provides an incredible opportunity to investigate 
initial stages of an online social network. The data was scraped by Reddit user \href{https://www.reddit.com/user/Stuck_In_the_Matrix/}{u/Stuck\_In\_the\_Matrix} in 2016 
and torrented from the \href{https://academictorrents.com/details/85a5bd50e4c365f8df70240ffd4ecc7dec59912b}{Academic Torrents} website. The original dataset includes 
comments up to at least 2015, but it would not have been feasible to include the whole of Reddit's comment dataset. However, the compressed dataset size is still 932 Mb, and 
covers the initial growth phase of the site, including the first year during which Reddit let users create their own subreddits. 
 
Each line in the decompressed dataset contains data about one comment and is organized as a JSON document. The relevant fields are as follows:
\begin{itemize}
    \item \code{subreddit} -- the subreddit under which the original link is;
    \item \code{subreddit_id} -- the ID of the subreddit; 
    \item \code{author} -- the username of the author;
    \item \code{body} -- the body of the comment;
    \item \code{score} -- the karma on the comment;
    \item \code{link_id} -- the ID of the link under which the comment was posted;
    \item \code{parent_id} -- the ID of the parent, which can be either the link or another comment;
    \item \code{id} -- the ID of the comment itself;
    \item \code{created_utc} -- the UNIX timestamp of the comment creation. 
\end{itemize}
There are more metadata, e.g. a field \code{controversiality}, which measures whether a comment has received a similar number of upvotes and downvotes, etc. Also, during the 
first years there were less fields (downvotes for example did not appear before 2008), so the selection was partly motivated by ensuring that all the fields would be present 
during the whole period under analysis. 

Some basic cleaning was applied to the data. First, the fields \code{link_id}, \code{parent_id} needed some preprocessing to be compatible and comparable with the field \code{id}.
Secondly, due to a relatively large number of comments authored by now-deleted users, some preliminary analysis steps required eliminating comments that were authored by 
the deleted users. This is because the name of the deleted comments in the Reddit's database is \code{"[deleted]"}, i.e. the statistics about user comment distributions, 
max number of posts, etc were skewed due to the there being a lot of comments authored by a deleted user (approximately 30\% of the comments). 


\includegraphics[width=0.4\textwidth]{subreddits.png}

From the figure above we can see that the number of subreddits grew rapidly and then exploded at the start of 2008 due to Reddit allowing users to create their own subreddits.
However, in subsequent analysis only the data up to March 2007 were used due to the volume of data.

\includegraphics[width=0.4\textwidth]{commenters.png}

The steady but fast growth of users is also clear from the data. 

\section{Methodology}

The methodology in this project follows the steps below. 
\begin{enumerate}
    \item Finding and downloading data.
    \item Preliminary preprocessing, ensuring the high quality of data.  
    \item Preliminary data analysis, investigating potentially interesting further research paths. 
    \item Create a co-commentator graph.
    \item Analyze time-evolution of network properties.
    \item Analyze the possibility of applying a link prediction model on the graph. 
\end{enumerate}

Co-commentator graph is formed as follows. Each comment contains a \code{link_id}, i.e. the post ID under which the post was made. All the comments under each post 
were collected and the authors of the comments were extracted. Then, each author was added to the graph as a node, and an edge was formed with all the other users 
who commented under the same post, i.e. with all the other co-commentators. 

Depending on the analysis, this was done separately for posts under a specific subreddit or for a specific time-period (generally a month). 


\section{Results}

The most interesting results were related to the first popular non-frontpage subreddits -- science and programming. It appeared that people who commented under those subreddits
had much higher centrality scores (eigenvector and PageRank), and kept commenting month-over-month. This is significant, since the comparative dataset did not include 
all users, but users who had already commented under a post with at least one other comment, i.e. the users who were not interested in the website and just signed up were 
excluded already. So the subreddits were clearly integral to the success in Reddit by drawing more engaging users and interactions. 

%\includegraphics[width=0.4\textwidth]{subreddits_activity.png}


\section{Conclusion}

% conference papers do not normally have an appendix


% optional entry into table of contents (if used)
%\addcontentsline{toc}{section}{Acknowledgment}


% trigger a \newpage just before the given reference
% number - used to balance the columns on the last page
% adjust value as needed - may need to be readjusted if
% the document is modified later
%\IEEEtriggeratref{8}
% The "triggered" command can be changed if desired:
%\IEEEtriggercmd{\enlargethispage{-5in}}

% references section
% NOTE: BibTeX documentation can be easily obtained at:
% http://www.ctan.org/tex-archive/biblio/bibtex/contrib/doc/

% can use a bibliography generated by BibTeX as a .bbl file
% standard IEEE bibliography style from:
% http://www.ctan.org/tex-archive/macros/latex/contrib/supported/IEEEtran/bibtex
%\bibliographystyle{IEEEtran.bst}
% argument is your BibTeX string definitions and bibliography database(s)
%\bibliography{IEEEabrv,../bib/paper}
%
% <OR> manually copy in the resultant .bbl file
% set second argument of \begin to the number of references
% (used to reserve space for the reference number labels box)
\bibliographystyle{IEEEtran}
\bibliography{final_report_bib}
%\begin{thebibliography}{1}


%\bibitem {cantrell1}
%W. H. Cantrell, ``Tuning analysis for the high-Q class-E power
%amplifier,'' \emph{IEEE Trans. Microwave Theory \& Tech.}, vol. 48,
%no. 12, pp. 2397-2402, December 2000.

%\bibitem {cantrell2}
%W. H. Cantrell, and W. A. Davis, ``Amplitude modulator utilizing a
%high-Q class-E DC-DC converter'', \emph {2003 IEEE MTT-S Int. Microwave
%Symp. Dig.}, vol. 3, pp. 1721-1724, June 2003.

%\bibitem {krauss}
%H. L. Krauss, C. W. Bostian, and F. H. Raab, \emph{Solid State Radio Engineering}, New York: J. Wiley \& Sons, 1980.

%%\bibitem{IEEEhowto:kopka}
%%H.~Kopka and P.~W. Daly, \emph{A Guide to {\LaTeX}}, 3rd~ed.\hskip 1em plus
%% 0.5em minus 0.4em\relax Harlow, England: Addison-Wesley, 1999.

%%\bibitem{lamport} L. Lamport, \emph{ {\LaTeX} A Document Preparation
%%  System}, Reading, Mass: Addison-Wesley, 1994.

%%\bibitem{knuth} D. E. Knuth, \emph {The \TeX book}, Reading, Mass.:
%%  Addison-Wesley, 1996.

%\end{thebibliography}
\smallskip
Note: For the Summary paper submission only, references to the authors own work should be cited as if done by others to enable a double-blind review. {\bfseries Citations must be complete and not redacted, allowing the reviewers to confirm that prior art has been properly identified and acknowledged.}
% that's all folks
\end{document}
